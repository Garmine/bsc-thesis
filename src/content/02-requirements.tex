%----------------------------------------------------------------------------
% 2. Követelmények megfogalmazása és ismertetése
%
% A feladatkiírás pontosítása és részletes elemzése

%\chapter{Introduction}
%\chapter{Earlier monitoring systems implemented in the dormitory}
\chapter{Requirements of the New Monitoring System \label{ch2}}
%\chapter{Available industry standard technologies}
%\chapter{System architecture}
%\chapter{Implementation details and experiences}
%\chapter{Results and evaluation}
%\chapter{Future development opportunities}
%----------------------------------------------------------------------------

Based on lessons learned from previous implementations and keeping in mind the
structure and aims of the \kszkfull we can define the hard and soft
requirements of a new and effective monitoring system.

%----------------------------------------------------------------------------
\section{Collecting and Storing Metrics and Logs}
%----------------------------------------------------------------------------

First of all, we have to collect various kinds of metrics logs from many kinds
of services and hardware devices. This is essential in achieving the main aims
of this project.

Hardware devices include but are not limited to:
\begin{itemize}
	\item servers running Windows Operating Systems
	\item servers running Linux Operating Systems
	\item servers running VMWare ESXi
	\item various Cisco switches, routers and access points
	\item other exotic hardwares such as the ones observing the washing machines and dryers of the dormitory
\end{itemize}

Services include but are not limited to:
\begin{itemize}
	\item various databeses: MySQL, PostgreSQL, etc.
	\item virtualization clusters: VMWare, Hyper-V, etc.
	\item the Kubernetes cluster
	\item storage servers and storage clusters
	\item the mailing service and mailing lists
	\item the dormitory's network and internet access
	\item unique services developed for internal use
\end{itemize}

Collecting metrics from higher level devices such as Linux hosts can be
achieved in many ways as these can run almost anything. The most common way of
achieving this is by running one or more monitoring daemons on the specific host.
These daemons expose the metrics via an HTTP interface from which the
monitoring solution pulls the metrics from.  As for network devices the options
are limited as often it is impossible to run arbitrary software on them. The
industry standard way of probing these devices is via the SNMP protocol.

Collecting logs on the other hand is the other way around: devices push their
logs onto the centralized log collector server. Servers and such have many
options to chose from just like before. The industry standard solution for
network devices is via the syslog protocol.

Collecting metrics and logs (if applicable) from unique devices and services
might require special solutions and thus the system must be highly flexible.

%----------------------------------------------------------------------------
\section{Effective long-term storage of historical data}
%----------------------------------------------------------------------------

Even though the dormitory's IT systems are small compared to a larger IT
company the sheer amount of metrics and logs collected from every device and
service can reach huge sizes on the long run. We must keep in mind that the
\kszk's resources are limited.

In order to minize the storage requirements the system must be capable of
selective retention of the collected data. Per-second CPU metrics don't hold
much value after a day or so while firewall logs might be valuable even after
years.

Older data is much less frequently accessed. The ability to store these entries
highly compressed and on another dedicated storage server would be much more
cost-effective and thus highly desired.

%----------------------------------------------------------------------------
\section{Detecting issues and alerting the necessary personnel}
%----------------------------------------------------------------------------

Through analyzation of the collected data we can identify coming and ongoing
issues. Automatic detection of these problems and the alerting of the right
\kszk members will prevent failures, improve the uptime and quality of our
services and thus keeping our users more happy.

Care must be taken though. False alerts or notifications sent to the wrong
maintainers will decrease confidence in the system or worse real malfunctions
are going to be ignored.

The \kszk's main internal communication platforms are via e-mail and
Mattermost. The system must be able to use these platforms at least.

%----------------------------------------------------------------------------
\section{Detecting issues of the monitoring system itself (meta-monitoring)}
%----------------------------------------------------------------------------

Relying on automated monitoring can make the life of \kszk's network and system
administrators much easier but this dependency has a major risk. In case parts
or whole of the monitoring system fails no automatic warnings will be
generated. Another issue is that the monitoring server is located in the
\kszk's server room inside the \schfull. In case the \sch's internet connection
is severed the system can not send out alerts.

Monitoring of the monitoring system itself (either by the system monitoring
itself or two systems monitoring each other) is called meta-monitoring.

Partial problems of the system (e.g. one effecting only the log collector) can
likely be detected by the system monitoring itself. On the other hand a
critical failure can only be detected by a separate, much simpler system
running offsite.

%----------------------------------------------------------------------------
\section{Ease of use yet highly customizable}
%----------------------------------------------------------------------------

Not only the \kszk's hardware resources are limited so are it's members time.
For this reason the configuration of metric scraping, log collection and
alerting must be easy and should not require any special knowledge from anybody
other than the monitoring system's adminitrator(s).

At the same time the system must be highly customizable to accomodate unique
solutions.

This can be achieved by using industry-standard solutions (eliminating a steep
learning curve) and fully automating common configurations while leaving custom
solutions possible.

%----------------------------------------------------------------------------
\section{Authentication and authorization via AuthSCH}
%----------------------------------------------------------------------------

The system thus must be able to authenticate and authorize it's users via OAuth
2.0.

Based on the user's AD groups:
\begin{itemize}
	\item monitoring system admins must have superadmin priviliges
	\item \kszk full members must be able to view everything and modify existing services or add new ones
	\item \kszk fresh recruits must be able to view every log and metric
	\item everyone else must be denied access
\end{itemize}

%----------------------------------------------------------------------------
\section{Visualization of collected data}
%----------------------------------------------------------------------------

All the collected data would be meaningless without proper interpretation and
visualization. Thus it is crucial for the system to provide a modern, web-based
and easy to use interface for visualizing collected metrics and browsing logs.
Users should be able to create new dashboards, access logs and much more
without the help of the monitoring admins.

Another important aspect of a pretty and modern interface are public relations.
With easy to grasp graphs, diagrams and numbers it's much simpler to explain
the countless hours the \kszk's voluntary administrators put into it's many
services. More importantly it could inspire the upcoming generation to take
part in our work and learn with us.

%----------------------------------------------------------------------------
\section{Security}
%----------------------------------------------------------------------------

As this system must have access to many internal systems and resources proper
security is crucial. Not only the system must be defended from outside harm, it
must also validate the authenticity of every metric and log it has access to.
Without proper measures an adversary could inject false information into the
system or much worse gain access to sensitive information.
