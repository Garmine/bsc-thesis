%----------------------------------------------------------------------------
% 0. Probléma ismertetése röviden
%   - Mi az a monitoring & alerting, miért jó nekünk
%
% Bevezetés: a feladat értelmezése, a tervezés célja, a feladat indokoltsága, a
% diplomaterv felépítésének rövid összefoglalása
%
% A bevezető tartalmazza a diplomaterv-kiírás elemzését, történelmi
% előzményeit, a feladat indokoltságát (a motiváció leírását), az eddigi
% megoldásokat, és ennek tükrében a hallgató megoldásának összefoglalását.
%
% A bevezető szokás szerint a diplomaterv felépítésével záródik, azaz annak
% rövid leírásával, hogy melyik fejezet mivel foglalkozik.

\chapter{Introduction \label{ch0}}
%\chapter{Earlier monitoring systems implemented in the dormitory}
%\chapter{Requirements of the new monitoring system}
%\chapter{Available industry standard technologies}
%\chapter{System architecture}
%\chapter{Implementation details and experiences}
%\chapter{Results and evaluation}
%\chapter{Future development opportunities}
%----------------------------------------------------------------------------

%----------------------------------------------------------------------------
\section{Monitoring Systems}
%----------------------------------------------------------------------------

Maintaining one or two servers for personal purposes is not a daunting task. We
can easily keep an eye on our hardwares and apply fixes when necessary. And if
for some reason a server goes down it is not likely to cause issues for others.

In a real-world production infrastructure the situation is radically different.
Manually checking each physical machine's state is an overwhelming burden and
it quickly becomes unfeasible. Sooner or later problems that could have been
prevented are going to cause service disruptions. This is going to cause issues
for users and the maintainers will only be notified through users' complaints.
At this point in time it is too late as the service's reputation is damaged and
the users are dissatisfied.

The best solution to this is an automated monitoring system. These sofware
stacks collect metrics from servers and other devices, detect error conditions
and try to predict future failures. If neccessary, the personnel in charge are
notified to take immediate action preventing the issue from affecting too many
users. Modern systems also support the collection of logs as well as advanced
analysis and visualization of all available data.

%----------------------------------------------------------------------------
\section{\kszkfull of \schfull}
%----------------------------------------------------------------------------

\kszkfull (\kszk for short) is a team of around 20 day-to-day active members.
They are volunteers from the students of \vik. They provide various IT
services for the Faculty's students and are responsible for the operation of
\schfull's whole network.

The monitoring system designed and implemented in this thesis will hopefully
help \kszk improve the quality of their services while reducing workload on
members.

\subsection{Infrastructure}

The IT infrastructure is comparable to that of a medium sized office building.
\kszk's Sysadmin group operates around 20 physical servers running
virtualization clusters, storage clusters and much more. \kszk's NETeam group
operates around 10 physical servers and 50 network devices serving more than
1000 endpoints for students living inside the \sch.

\subsection{AuthSCH: Authentication and Authorization}

AuthSCH is the central authentication and authorization service. It was created
by Zoltán Janega \cite{ZolijBsc}, a veteran member of \kszk to replace the 3
authentication systems used back then. Users data is stored in Active Directory
and authentication is done over the OAuth 2.0 protocol.

\subsection{osTicket: Support Ticketing System}

Users' issues are reported through an osTicket instance. Between 20 to 100
tickets are opened in each month, depending on which part of a semester we are
in. Answering and resolving these tickets take a lot of time.

\subsection{Mattermost: Main Communication Platform}

Mattermost is an open-source communication platform with a rich feature set.
Communications between the members of \kszk mainly take place on a self-hosted
Mattermost server besides e-mails.

\subsection{GitLab: DevOps Platform}

GitLab is an open-core DevOps platform. It's not only a git server as it
features an issue tracker, Continus Integration and much more. \kszk operates a
self-hosted GitLab server for storing their source codes and configurations and
to run CI jobs and more.

%----------------------------------------------------------------------------
\section{Overview of this Thesis}
%----------------------------------------------------------------------------

Monitoring systems are not unknown to \kszk. In \autoref{ch1} we are going to
study earlier takes on implementing such systems in the \sch to figure out what
is needed for a successful implementation.

In \autoref{ch2} we are going to define the hard requirements for a monitoring
system in \kszk's infrastructure.

After that we are going to explore modern technologies suitable for our needs
in \autoref{ch3}.

Using the results of our research in previous chapters we are going to describe
the new system architecture in \autoref{ch4}.

In \autoref{ch5} we are going to start implementing the specified architecture
and work out the details. 

We are going to evaluate the new system in \autoref{ch6} and check out the
results.

Finally, in \autoref{ch7} we are going to discuss future work including
possible improvements and new ideas as well.
