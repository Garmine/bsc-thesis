%----------------------------------------------------------------------------
% 6. A futó rendszer értékelése
%   - Összevetés a követelményekkel
%   - Pro/contra, jó és rossz tapasztalatok
%
% A megtervezett műszaki alkotás értékelése, kritikai elemzése

%\chapter{Introduction}
%\chapter{Earlier monitoring systems implemented in the dormitory}
%\chapter{Requirements of the new monitoring system}
%\chapter{Available industry standard technologies}
%\chapter{System architecture}
%\chapter{Implementation details and experiences}
\chapter{Results and evaluation \label{ch6}}
%\chapter{Future development opportunities}
%----------------------------------------------------------------------------

During the time of writing the monitoring system has been running for months
now with gradually increasing capabilities as time went on. Although most
requirements were met the system is still in development. Let's take a look at
the current state of it.

%----------------------------------------------------------------------------
\section{Evaluation of Requirements Fulfillment}
%----------------------------------------------------------------------------

The most important is meeting the requirements laid out in \autoref{ch2}.

\subsection{Collecting and storing metrics and logs}

The currently installed system is more than capable of collecting and storing
metrics and logs. We are able to monitor every server platform and network
device currently in use by \kszk.

At the time of writing many systems including Windows Servers, Linux Servers
and Cisco devices are already being monitored. Only logs are not being
collected from network devices due to debates about an optimal and secure
solution.

\subsection{Effective long-term storage of historical data}

The system is currently capable of storing data for short to mid terms. The
implementation of long-term storage was least concern. With proper
configuration such as fine tuned retentions and external storage both
VictoriaMetrics and Loki are able to effectively storage large amounts of data
only limited by available storage.

\subsection{Detecting issues and alerting the necessary personnel}

VictoriaMetrics' components already detect some basic issues and send out
alerts to a test channel on \kszk's Mattermost platform. The software stack is
capable of detecting much more complex issues and sending out alerts on many
communications channels. This perfectly demonstrates our top priority:
preventing future issues and detecting existing ones.

\subsection{Detecting issues of the monitoring system itself (meta-monitoring)}

Meta-monitoring is partially implemented. The system collects and visualizes
many metrics and logs about itself as well has many alerts configured to notify
it's maintainers about intervention needed. There are still some places to
improve though. External monitoring of the system is very much possible but is
not yet implemented at all. This is a crucial next step for production.

\subsection{Ease of use yet highly customizable}

The addition of CI and configuration generation made maintenance and the
connecting of new systems very easy. The usage of industry-standard tools such
as Kubernetes and Helm Charts and simple data formats such as YAML files help
make the learning curve shallow. Internal documentations as well as this paper
also help users and administrators alike.

\subsection{Authentication and authorization via AuthSCH}

Authentication and authorization via AuthSCH using OAuth 2.0 protocol is fully
implemented and is already being used by \kszk's members.

\subsection{Visualization of collected data}

Visualization of collected logs and metrics is done using Grafana, a leading
industry-standard tool. The vast array of publicly available dashboards and the
easy configuration of new ones means it is able to handle everything collected
by the rest of the software stack. The configured UI already handles the most
important and interesting data correctly and presents them in an easy to
understand graphical way.

\subsection{Security}

Access to the system is only granted to \kszk's members, the server is equipped
with up to date software, an open source version of an enterprise Linux
distribution, protected by firewall and connection can only be made through
TLS. Detailed security analysis was not yet done as some features still have to
be improved upon.

\subsection{Summary}

All in all the new monitoring system already fully satisfies 4 and partially
satisfies 3 out of the 8 main requirements. There is no known difficulty in
fully satisfying every requirement.

%----------------------------------------------------------------------------
\section{Evaluation of the Performance of the System}
%----------------------------------------------------------------------------

\subsection{CPU and RAM Usage}

Currently system load and CPU usage is around 4\% on average. RAM usage is
12GBs out of 32GB totally installed

VictoriaMetrics scales very well even with much larger amounts of collected
metrics.

Loki's memory usage depends on it's configrations (e.g. cluster or monolithic
modes) and the types of queries performed. The current setup uses small amount
of memory when executing simpler queries. The amount of available RAM shouold
be fine even with more hosts. For more complex queries with a larger number of
hosts fine tuning of Loki is going to be neccessary and switching to a cluser
mode of Loki is advised.

\subsection{Disk Usage}

At the moment only 13\% of the SSDs are being used by the host system. This is
not expected to change much in the future as both metrics and logs are stored
on the HDDs.

Out of the 1TB available space on the ZFS pool located on the HDDs around 5GB
is already being used by metrics and logs collected. This is after around one
month of continous data collection. At this rate the HDDs will be fully
utilized in 17 years. Right now at most 50\% of available metrics and 10\% of
available logs are being collected. With larger adaptation this would only last
around one to two years.

Considering this during the following year data retention policies will have to
be fine tuned and new, larger capacity HDDs will have to be acquired. A
long-term and more effective storage solution should the developed in the next
two years.

\subsection{Network Usage}

Network traffic is very minimal. The server has 4x1Gbit ethernet ports yet we
only use around 4Mbit/s on the single connected port. Even with all ports
redundantly configured and 500 times more incoming data the network bandwidth
will not be a bottleneck.

\subsection{Summary}

The currently configured storage is ideal for development and testing purposes
but it's going to be the first issue mid-term. This is very much expected from
a service that collects data in every millisecond in it's uptime. For \kszk's
infrastructure size CPU and RAM may not be a problem but fine tunings and
configuration changes will have to be made on the long run.

%----------------------------------------------------------------------------
\section{Results of Continous Integration}
%----------------------------------------------------------------------------

\subsection{CI for Collecting Metrics from Servers}

Before CI only 3 server were monitored for demonstration purposes. After the
release of CI features to \kszk's member they quickly and voluntarily started
to connect their own systems with only minimal to no help other than the
provided documentation.

Systems added by our users through CI:
\begin{itemize}
	\item 11 Linux systems with 19 exporters in total
	\item 6 Windows systems with 6 exporters in total
\end{itemize}

Exporters installed by our users include from the generic Linux/Windows node
exporter to more specific ones such as database and ZFS exporters. They also
imported or created many new dashboards when it was neccessary.

\subsection{CI for Collecting Metrics from Network Devices}

CI was also implemented for network devices and our users made use of it. After
release 11 more Cisco devices were added to the system similarly to the
previous case. A basic SNMP dashboard was also released that was later extended
and improved by members of \kszk.

\subsection{CI for Configuring Alerts}

No new alerts were configured yet. The obvious cause of this is a lack of time
as alerts were introduced recently compared to the time of writing. Alerts'
configuration generation is also much less automated than the previously
reviewed ones.

\subsection{Summary}

The three cases discussed above prove that with a carefully designed and easy
to use configuration generation system members of \kszk will voluntarily
connect their systems into the new monitoring apparatus with minimal or no help
from it's maintainers.

%----------------------------------------------------------------------------
\section{Other Results of the Monitoring System running so far}
%----------------------------------------------------------------------------

The self-monitoring capabilities already helped during some experimentation and
debug sessions.

The preconfigured alerts managed to find two NTP synchronization issues
immediately after only monitoring was configured. After the alert manager was
properly configured later on we instantly received notifications for these
alerts on \kszk's Mattermost.

