%----------------------------------------------------------------------------
% 3. Lehetséges (új/modern) megoldások vizsgálata, elemzése és összehasonlítása
%   - Rövid történet, kik írták, kik használják, mire használják
%   - Rövid, elméleti "mi lenne ha használnánk" kifejtés
%   - Összehasonlítás, döntés
%
% Előzmények (irodalomkutatás, hasonló alkotások), az ezekből levonható
% következtetések
% A tervezés részletes leírása, a döntési lehetőségek értékelése és a
% választott megoldások indoklása

%\chapter{Introduction}
%\chapter{Earlier monitoring systems implemented in the dormitory}
%\chapter{Requirements of the new monitoring system}
\chapter{Available industry standard technologies \label{ch3}}
%\chapter{System architecture}
%\chapter{Implementation details and experiences}
%\chapter{Results and evaluation}
%\chapter{Future development opportunities}
%----------------------------------------------------------------------------

In this chapter we are going to view and compare some technologies that are
able to satisfy our requirements.

%----------------------------------------------------------------------------
\section{Prometheus vs VictoriaMetrics}
%----------------------------------------------------------------------------

\subsection{Prometheus \label{bbPrometheus}}

Prometheus\cite{prom} is an open-source monitoring system that is designed to monitor the
performance and health of a computer system. It is commonly used in large,
complex environments and is known for its scalability, reliability, and ability
to provide real-time alerts and notifications when problems arise.

\subsection{VictoriaMetrics}

VictoriaMetrics\cite{vm} is an open-source time-series database and a monitoring
solution that is designed to be highly efficient and scalable. Additionally, it
offers a number of features that make it easy to manage and maintain, such as
automatic data retention and garbage collection.

\subsection{Comparison}

Both systems are capable of performing the needed tasks. As VictoriaMetrics
uses and/or compatible with Prometheus APIs it can work with the exact same
software environments and can be a drop-in replacement. This is handy for us as
\kszk's system administrators are already familiar with the needed exporters.

VictoriaMetrics has some clear advantages over Prometheus for us\cite{vmspeed}:
\begin{itemize}
	\item it's local storage can work as a durable long-term storage % https://tech.bedrockstreaming.com/2022/09/06/monitoring-at-scale-with-victoriametrics.html
	\item it utilizes CPU and RAM better
\end{itemize}

\subsection{Verdict}

Although \kszk had utilized Prometheus in the past we decided to use
VictoriaMetrics for this project.

%----------------------------------------------------------------------------
\section{ELK stack vs Loki and Promtail}
%----------------------------------------------------------------------------

\subsection{ELK stack \label{bbElk}}

An ELK stack\cite{elk} is a term used to refer to a combination of three open-source
tools: Elasticsearch, Logstash, and Kibana. These tools are used for different
purposes, but they are often used together to create a powerful logging and
monitoring solution. Elasticsearch is a search engine and data analysis tool
that is used to store and index large volumes of data. Logstash is a data
collection and processing tool that is used to ingest, transform, and filter
data from a variety of sources. Kibana is a data visualization tool that is
used to create dashboards and other visualizations from the data stored in
Elasticsearch.

\subsection{Loki and Promtail}

Loki and Promtail\cite{lp} are part of a set of open-source tools for log management.
Loki is a horizontally-scalable, highly-available log aggregation system. It is
designed to be very efficient and can handle large amounts of log data.
Promtail is a log scraping agent that can collect log data from a variety of
sources, including containers and Kubernetes environments. It is designed to
work seamlessly with Loki, allowing you to easily ingest and query your log
data.

\subsection{Comparison}

ELK and Loki serve the same purpose yet they are completely different.

Advantages of Loki over ELK:
\begin{itemize}
	\item easy to operate
	\item requires much less resouces
	\item collected logs can be visualized by Grafana
	\item authorization (through Grafana)
\end{itemize}

Advantages of ELK over Loki:
\begin{itemize}
	\item supports much more complex searches
\end{itemize}

\subsection{Verdict}

For our use cases Loki's capabilities seem fine although we can not see the
future. On the other hand an ELK stack requires huge amounts of resources which
would be a problem soon. We decided to use Loki.

%----------------------------------------------------------------------------
\section{Grafana \label{bbGrafana}}
%----------------------------------------------------------------------------

Grafana\cite{g} is an open-source data visualization tool known for its powerful and
customizable dashboard features, which allow users to create real-time graphs,
charts, and other visualizations from the data collected by other tools.

Grafana supports a large array of data sources including Prometheus,
VictoriaMetrics and Loki. It also supports authentication and authorization
through OAuth 2.0.
