%----------------------------------------------------------------------------
% 7. Jövőbeni teendők, hogyan lehetne továbbfejleszteni a rendszert
%
% továbbfejlesztési lehetőségek

%\chapter{Introduction}
%\chapter{Earlier monitoring systems implemented in the dormitory}
%\chapter{Requirements of the new monitoring system}
%\chapter{Available industry standard technologies}
%\chapter{System architecture}
%\chapter{Implementation details and experiences}
%\chapter{Results and evaluation}
\chapter{Future Development Opportunities \label{ch7}}
%----------------------------------------------------------------------------

Besides the smaller imperfections and missing features described in the
previous chapter there are still many great ideas that could be implemented in
the future. Some of these are detailed in this chapter.

%----------------------------------------------------------------------------
\section{Improving Configuration Management and Enforcing Reproducibility}
%----------------------------------------------------------------------------

During development it was easier to take some shortcuts. Unfortunately this has
resulted in violation of reproducibility. This must be addressed in the future.

\subsection{Grafana Configurations}

Grafana has a rich UI where users can configure and edit many things. This also
means that any changes made are stored in some format internally. Currently
these would be lost during a reinstallation. Applying some sort of version
control to these would solve this problem.

\subsection{Backups}

Currently there are no backups of the system. In case of a catastrophic failure
data and configurations could be lost forever which is unacceptable. Regular
automated backups are the solution.

\subsection{Ansible Scripts}

Ansible scripts might be out of sync currently. Regurarly verifyng the
reproducibility of installation and configuration would ensure the integrity of
these. These, together with backups would make recoveries and hardware upgrades
very quick and painless.

\subsection{Advanced Configuration Generation \label{advconf}}

Currently the three main configurations (exporters, SNMP and alerts) are
completely separate. With the development of some common data model and
configuration generation these could be unified. Adding a library of common
configurations (such as alerts) would make the addition of new systems seem
like child's play.

%----------------------------------------------------------------------------
\section{Granting Users Access to Internal Components}
%----------------------------------------------------------------------------

Many information about the system's internal states are available to our users
thanks to self-monitoring. Yet there are still some components that would be
very useful to them for debugging and experimentation.

These are not available to users at the moment because authentication and
authorization is not yet solved for them.

\subsection{Access to the VMAgent component}

VMAgent has a web UI where status and informations are displayed about every
exporter. In some cases VMAgent has access to a specific exporter yet it's
metrics are not seen by Grafan. This would help rule out communication issues
between exporters and VictoriaMetrics.

\subsection{Access to the VMAlert component}

VMAlert has an HTTP interface where every fired alert is sent in a JSON
response. This would help identify if a mistake was made in the VMAlert or the
VMAlertmanager configuration.

\subsection{Access to the VMAlertmanager component}

VMAlertmanager has a web interface where every information is available about
fired alerts and some actions (e.g. silencing) can be taken. This would be
extremely useful for debugging alert routing and notifications.

%----------------------------------------------------------------------------
\section{Improving Log Collection}
%----------------------------------------------------------------------------

Log collection is very bare-bones right now and there is much room for
improvement.

\subsection{Configurations for Remote Promtails}

Installing promtails on remote servers is the lone duty of the remote sysadmin
right now. User experience could be greatly improved by providing extra
documentation and examples for our users.

\subsection{Collecting Logs from Network Devices}

At the time of writing no logs are being collected from network devices. An
architecture similar to the SNMP exporters' could be a solution to this
problem.

%----------------------------------------------------------------------------
\section{Advanced Log Analysis}
%----------------------------------------------------------------------------

Once the monitoring system has access to every metric and log it's going to
have more than enough information to theoretically be able to detect complex
problems and suggest possible solutions.

These issues are usually detected by our users themselves and then are reported
in \kszk's support ticketing system. This is time consuming for both parties
and can be a bad user experience.

Advanced log analysis could solve this and provide users with enough
information to fix issues themselves preventing ticket creation.

Based on a crude analysis of tickets received during October 2022:
\begin{itemize}
	\item 32 tickets were received in total
	\item 12 tickets could potentially be solved with this method
	\item  8 tickets would need the assistance of a member of \kszk but we could create alerts and/or provide useful information
	\item 11 tickets were requests that could be automated but not with the monitoring system
	\item  1 ticket  could not be categorized
\end{itemize}

The second category is the one we are aiming for. A typical ticket of this
category is about having no internet access. The \sch's network denies access
to unknown devices. Often the cause of this is MAC randomization which could
easily be detected by analysing SNMP and/or network logs.

%TODO example ticket screenshot

%----------------------------------------------------------------------------
\section{Service Discovery}
%----------------------------------------------------------------------------

With Advanced Config Generation (\autoref{advconf}) adding new systems could be
easy. But with service discovery this task could be (almost?) completely
automated. There's nothing easier than doing nothing.

VictoriaMetrics supports many kinds of service discoveries. These methods
should be explored and evaluated for a possible solution.

%----------------------------------------------------------------------------
\section{Other Possibilities}
%----------------------------------------------------------------------------

A few ideas that could be improved upon:
\begin{itemize}
	\item Loki self-monitoring using Loki Canary
	\item Interactive messages for Mattermost
	\item Public vizualizations without authentication
\end{itemize}
