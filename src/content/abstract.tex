\pagenumbering{roman}
\setcounter{page}{1}

\selecthungarian

%----------------------------------------------------------------------------
% Abstract in Hungarian
%----------------------------------------------------------------------------
\chapter*{Kivonat}\addcontentsline{toc}{chapter}{Kivonat}

Az automatikus felügyelő rendszerek feladata metrikákat gyűjteni hardverekről
és szolgáltatásokról, hibákat találni és megpróbálni előre megjósolni a jövőben
bekövetkező meghibásodásokat. Amennyiben szükséges értesítik a megfelelő
személyzetet, hogy gyorsan intézkedhessenek ezáltal megakadályozva, hogy a
probléma túl sok felhasználóra legyen hatással. A modern rendszerek támogatják
továbba a naplóbejegyzések gyűjtését és fejlett elemzését, illetve minden
elérhető adat vizualizációját.

A Kollégiumi Számítástechnikai Kör (röviden KSZK) aktív tagsága nagyjából 20
fő. Ők a Villamosmérnöki és Informatikai Kar olyan hallgatói, akik önkéntesként
teszik munkájukat. Ők különféle informatikai szolgáltatásokat üzemeltetnek a
VIK hallgatói számára és emellett a Schönherz Zoltán Kollégium teljes hálózatát
is üzemeltetik.

A szakdolgozatomban egy automatikus felügyelő rendszer megvalósítását mutatom
be. Az elkészült rendszerben VictoriaMetrics gyűjti a metrikákat a szerverekről
és a hálózati eszközökről egyaránt. A VMAlertmanager komponens küldi a
riasztási értesítéseket egy Mattermost szerverre. Loki és Promtail gyűjti be és
dolgozza fel a naplóbejegyzéseket. Folyamatos Integrációt használok új
felügyelési célpontok bekötésére a futó rendszerbe úgy, hogy egy felügyelő
rendszer adminisztrátor közbenjárására sincs szükség.

Ez az új rendszer remélhetőleg segíteni fog a KSZK-nak emelni szolgáltatásai
színvonalát miközben csökkenti a tagjaira hárult terheket.

Ebben a dolgozatban leírom a teljes folyamatot, melynek eredménye a fent leírt
rendszer implementálása. Tanulmányozom a KSZK által korábban megvalósított
rendszereket annak érdekében, hogy rájöjjek mi kell egy sikeres
megvalósításhoz. Ezen eredmények alapján leírom a KSZK-ban futó felügyelő
rendszer kemény követelményeit és felkutatok olyan modern technológiákat,
amelyek ki tudják elégíteni ezen követelményeket. Az így szerzett tudást
felhasználva meghatározom egy VictoriaMetrics alapú rendszer szerkezetét, majd
elkezdem kidolgozni a részleteket és megvalósítani a tervet. Értékelem az új
rendszert és ellenőrzöm az elért eredményeket. Végül kifejtem miket lehetne még
tenni a jövőben, beleértve lehetséges továbbfejlesztéseket és teljesen új
ötleteket is.

A dolgozat során elkészült rendszer megfelel a felállított követelményeknek. A
KSZK tagjai már használatba is vették az elkészült funkciókat és a Folyamatos
Integráció segítségével sok új rendszert kezdtek el megfigyelni.


\vfill
\selectenglish


%----------------------------------------------------------------------------
% Abstract in English
%----------------------------------------------------------------------------
\chapter*{Abstract}\addcontentsline{toc}{chapter}{Abstract}

Automated monitoring systems collect metrics from hardwares and services,
detect error conditions and try to predict future failures. If neccessary the
right personnel are notified to take immediate action preventing the issue from
affecting too many users. Modern systems also support the collection of logs as
well as advanced analysis and visualization of every available data.

\kszkfull (\kszk for short) is a team of around 20 day-to-day active members.
They are volunteers from the students of \vik. They provide various IT services
for the Faculty's students and are responsible for the operation of \schfull's
whole network.

In this thesis I implement an automated monitoring system. VictoriaMetrics
scrapes metrics from servers and network devices. VMAlertmanager sends alert
notifications to a Mattermost server. Loki and Promtail collect and process
logs. Grafana visualizes collected metrics and logs. Continous Integration is
used to integrate new monitoring targets into the running system without the
intervention of a monitoring system administrator.

This new system will hopefully help \kszk improve the quality of their services
while reducing workload on members.

In this thesis I describe the whole process resulting in the implementation of
the system described above. I study earlier takes on implementing such systems
by \kszk to figure out what is needed for a successful implementation. Based on
these results I establish the hard requirements for a monitoring system in
\kszk's infrastructure and explore modern technologies that can satisfy these
requirements. Using the knowledge gained I describe the new system architecture
for a VictoriaMetrics based solution and implement the specified architecture
and work out the details. I evaluate the new system and check out the results.
Finally I future work including possible improvements and new ideas as well.

The implemented system satisfies the requirements established in this thesis.
Members of \kszk are already using the completed functions and started
monitoring various new systems with Continous Integration.


\vfill
\cleardoublepage

\selectthesislanguage

\newcounter{romanPage}
\setcounter{romanPage}{\value{page}}
\stepcounter{romanPage}
